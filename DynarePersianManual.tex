\documentclass[12pt]{article} 
\usepackage[margin=2cm]{geometry} 
\usepackage{psfrag} 
\usepackage{graphicx} 
\usepackage{pgfplots} 
\pgfplotsset{compat=newest} 
\usetikzlibrary{plotmarks} 
\usetikzlibrary{arrows.meta} 
\usepgfplotslibrary{patchplots} 
\usepackage{grffile} 
\usepackage{url}  
\usepackage{epstopdf} 
\usepackage{longtable,booktabs} 
\usepackage{amsmath,amsfonts} 
\usepackage{breqn} 
\usepackage{authblk}
\usepackage{float,morefloats,caption} 
\usepackage{xepersian} 
\settextfont{XB Zar} 
\setlatintextfont{Times New Roman} 
\setdigitfont{Yas} 
\title{فارسی‌ساز نرم‌افزار Dynare}
\author{حسین توکلیان\thanks{tavakolianh@gmail.com} }%\date{}
\affil{عضو هیئت علمی دانشکده اقتصاد دانشگاه علامه طباطبائی}
\begin{document} 
\maketitle

\section{مقدمه}
با توجه به این‌که نرم‌افزار Dynare در فضای نرم‌افزار MATLAB استفاده شده و بسیاری از کاربران از آن برای کارهای تحقیقاتی خود استفاده می‌کنند و نیاز به تبدیل خروجی‌های این نرم‌افزار به فارسی دارند، تغییرات ارائه شده در این بسته این نیاز را برطرف می‌سازد. برای این‌که بتوان از این بسته به‌درستی استفاده کرد به تغییراتی در سیستم عامل وجود دارد که در ادامه به این تغییرات اشاره خواهد شد. قبل از آن باید نرم‌افزارهای دیگری که مکمل این بسته هستند نیز باید معرفی گردد. این بسته قابل استفاده در دو سیستم عامل ویندوز و مک است. 
\section{نرم‌افزارهای مورد نیاز}
با توجه به این‌که Dynare به راحتی خروجی‌ها را به‌صورت لاتک ارائه می‌کند، اولین نرم‌افزاری که کاربر باید در دستگاه خود نصب کند نرم‌افزار ‪\LaTeX‬‬ است. با توجه به این‌که این نرم‌افزار یک نرم‌افزار ‪\lr{Open Source}‬‬‬ است، ممکن است کاربر با نسخه‌های مختلفی از آن روبرو شود. بنابراین در این‌جا توصیه اکید می‌شود از نرم‌افزار Protext استفاده شود که از لینک ‪\lr{\url{https://www.tug.org/protext/}}‬‬‬ نرم‌افزار ‪\lr{MiKTeX}‬‬ برای ویندوز و از لینک ‪\lr{\url{http://www.tug.org/mactex/}}‬‬‬ نرم‌افزار ‪\lr{MacTeX}‬‬‬ را برای مک دانلود کنند. برای استفاده از ‪\lr{Editor}‬‬‬ مناسب برای سیستم عامل ویندوز، در لینک مربوط به ویندوز و به همراه ‪\lr{MiKTeX}‬‬ ‬‬ نرم‌افزار ‪\lr{TeXstudio}‬‬ ‬‬‪  نیز وجود دارد که سازگار با نیم‌فاصله در کیبورد استاندارد فارسی است. در مورد سیستم عامل مک، هر ‪\lr{Editor}‬‬ ‬‬‬مد نظر کاربر با کلیه ویژگی‌های کیبورد استاندارد فارسی هم‌خوانی دارد اما نویسنده نرم‌افزار ‪\lr{TeXmaker}‬‬ ‬‬‬را توصیه می‌کند که از لینک ‪\lr{\url{http://www.xm1math.net/texmaker/}}‬‬‬ قابل دانلود است.

برای این‌که اصول نگارش فارسی به‌درستی رعایت شود، اکیداً توصیه می‌شود که فونت‌های سری ‪\lr{XB}‬‬ دانلود و در سیستم‌ عامل نصب شود چرا که فونت‌های سری ‪\lr{B}‬‬ دارای مشکلاتی است که بسیاری از اصول نگارش فارسی مانند تفاوت عدد ۵ در فارسی و عربی یا تفاوت حرف ی و ي را رعایت نمی‌کند. با جست‌وجوی ساده‌ای می‌توان فونت‌های سری ‪\lr{XB}‬‬ از سایت‌های مختلف دانلود کرد. بسته DynarePersian از فونت ‪\lr{XB Zar}‬‬‬ برای متن فارسی استفاده می‌کند. بنابراین اگر کاربر این فونت را در سیستم عامل خود نداشته باشد، قادر به دریافت خروجی فارسی نخواهد بود مگر این‌که در فایل‌های خروجی ‪\LaTeX‬‬‬ فونت را تغییر دهد. همچنین برای این‌که صفر فارسی که توخالی است (‪$0$‬‬‬‬) رعایت شود، حتماً باید فونت Yas در سیستم عامل نصب باشد. این فونت، فونت اعداد در بسته حاضر است. این دو فونت کلیدی در بسته ارائه شده وجود دارد.

آخرین جعبه‌ابزاری که قبل از استفاده از DynarePersian باید در سیستم عامل وجود داشته باشد جعبه‌ابزار بسیار قوی matlab2tikz است. tikz بسته‌ای است در ‪\LaTeX‬‬ که به کاربر امکان رسم اشکال با کیفیت و دقت بسیار بالا را مهیا می‌سازد. جعبه‌ابزار فوق جعبه‌ابزاری است که نمودارهای MATLAB را به tikz تبدیل می‌کند. دلیل این‌که از این جعبه‌ابزار در DynarePersian استفاده شده این است که کیفیت نمودارها با استفاده از این جعبه‌ابزار بسیار بالا خواهد بود اما مهمتر از آن این است که با استفاده از این ابزار اعداد نمودارها فارسی شده و امکان لحاظ نمادهای مختلف مانند حروف یونانی در نمودارها نیز مهیا خواهد شد. این جعبه ابزار را می‌توان از لینک ‪\lr{\url{https://github.com/matlab2tikz/matlab2tikz}}‬‬‬‬ دانلود کرد. پس از دانلود و unzip کردن آن در یکی از درایوهای دستگاه خود همانند Dynare باید مسیر آن را از طریق گزینه ‪\lr{Set Path}‬‬‬‬‬ نرم‌افزار MATLAB تنظیم کرد تا MATLAB بتواند از آن استفاده کند. ‪
\section{•}
\end{document}